\documentclass[tikz]{standalone}
\usepackage{fontspec}
\renewcommand*{\familydefault}{\sfdefault}
\usepackage{standalone}
\usetikzlibrary{arrows.meta, decorations.pathreplacing, shapes.geometric}
\usetikzlibrary{bayesnet}
% for phylogenies
\usetikzlibrary{graphs}
\usetikzlibrary{graphdrawing}
\usetikzlibrary{quotes}
\usegdlibrary{trees}
\usegdlibrary{phylogenetics}

\begin{document}

\begin{tikzpicture}


% plate: branch-homogeneity
\begin{scope}

\path (0,0)
node[obs] (X) {\(X^v_i\)}
(X) +(-0.7,-1) coordinate (site)
(X) +(90:1.0 cm) node[factor] (Q) {}
(Q) +(-135:0.4 cm) node (Q2) {\(Q^v\)}
(Q) +(0:1.75 cm) coordinate (tree)
(X) +(0:1.5 cm) node[latent] (fsel) {\(f^v_\mathrm{sel}\)}
;

\draw[->] (Q) -- (X);

% Plates
\plate {Xtree} {(X) (Q2) (fsel) (tree)} {\(v\in\mathcal{V}_\mathcal{T}\)} ;
\plate {Xsite} {(X) (site)} {\(i\in\mathcal{V}_\mathcal{G}\)} ;

\begin{scope}[font=\bfseries, pos=0.4, text=red]
\draw[dashed, red, -Stealth] (fsel) to[bend right] node[pos=0.4, fill=white] {?} (Q);
\path[anchor=south] (fsel) +(90:0.3 cm) node {?} ;
\end{scope}

\path (Q) +(0.5,0.25) node[anchor=south] {ág-heterogén};

\end{scope}


% plate: branch-homogeneity
\begin{scope}[yshift=-4 cm]

\path (0,0)
node[obs] (X) {\(X^v_i\)}
(X) +(-0.7,-1) coordinate (site)
(X) +(90:1.0 cm) node[factor] (Q) {}
(Q) +(180:0.4 cm) node (Q2) {\(Q\)}
(X) +(0:1.5 cm) coordinate (tree)
(X) +(0:2.5 cm) node[latent] (fsel) {\(f_\mathrm{sel}\)}
;

\draw[->] (Q) -- (X);

% Plates
\plate {Xtree} {(X) (tree)} {\(v\in\mathcal{V}_\mathcal{T}\)} ;
\plate {Xsite} {(X) (site)} {\(i\in\mathcal{V}_\mathcal{G}\)} ;

\begin{scope}[font=\bfseries, pos=0.4, text=red]
\draw[dashed, red, -Stealth] (fsel) to[bend right] node[pos=0.4, fill=white] {?} (Q);
\path[anchor=south] (fsel) +(90:0.3 cm) node {?} ;
\end{scope}

\path (Q) +(0.5,0.25) node[anchor=south] {ág-homogén};

\end{scope}


% phylogeny
\begin{scope}[line width=1 pt, xshift=20, yshift=10]

% phylogeny
\begin{scope}[xshift=4 cm, xscale=0.75, yscale=0.2, gray]
\node[black, anchor=west] at (-1,5) {lassú, globális adaptáció};
\graph[rooted straight phylogram, phylogenetic tree layout,
phylogenetic tree by author, grow=right,
nodes={font=\tiny, draw, fill, circle, inner sep=0 pt, minimum size=3 pt, as=}
]
{
9 -- {
8 [>length=2.5, gray!80!red!80!blue, >gray!80!red!80!blue] -- { 5[>length=1.5,
gray!80!red!70!blue, >gray!80!red!70!blue]
--[gray!80!red!60!blue] {c1[>length=2.5, gray!80!red!60!blue],
c2[gray!80!red!60!blue]}, 4[>length=1.8, gray!70!red!80!blue,
>gray!70!red!80!blue] --[gray!60!red!80!blue] { a1[>length=3,
gray!60!red!80!blue], a2[gray!60!red!80!blue] --[gray!50!red!80!blue]
{b1[>length=1.3, gray!50!red!80!blue], b2[>length=1.7, gray!50!red!80!blue]}} },
7 [>length=2.4, gray!80!brown, >gray!80!brown] --[gray!70!brown] {
6[>length=1.1, gray!70!brown] --[gray!60!brown] {3[>length=1.8,
gray!60!brown], 2[gray!60!brown]}, 1[>length=3.2, gray!70!brown]}
}
};
\end{scope}

% phylogeny
\begin{scope}[xshift=4 cm, yshift=-2.5 cm, xscale=0.75, yscale=0.2, gray]
\node[black, anchor=west] at (-1,5) {gyors, lokális adaptáció};
\graph[rooted straight phylogram, phylogenetic tree layout,
phylogenetic tree by author, grow=right,
nodes={font=\tiny, draw, fill, circle, inner sep=0 pt, minimum size=3 pt, as=}
]
{
9 -- {
8 [>length=2.5] -- { 5[>length=1.5] -- {c1[>length=2.5], c2}, 4[>length=1.8,
red]
--[red] { a1[>length=3, red], a2[red] --[red]
{b1[>length=1.3, red], b2[>length=1.7, red]}} },
7 [>length=2.4] -- { 6[>length=1.1] -- {3[>length=1.8], 2}, 1[>length=3.2]}
}
};
\end{scope}

% phylogeny
\begin{scope}[xshift=4 cm, yshift=-5.0 cm, xscale=0.75, yscale=0.2, gray]
\node[black, anchor=west] at (-1,5) {neutrális evolúció};
\graph[rooted straight phylogram, phylogenetic tree layout,
phylogenetic tree by author, grow=right,
nodes={font=\tiny, draw, fill, circle, inner sep=0 pt, minimum size=3 pt, as=},
]
{
9 -- {
8 [>length=2.5] -- { 5[>length=1.5] -- {c1[>length=2.5], c2}, 4[>length=1.8] --{ a1[>length=3], a2 --
{b1[>length=1.3], b2[>length=1.7]}} },
7 [>length=2.4] -- { 6[>length=1.1] -- {3[>length=1.8], 2}, 1[>length=3.2]}
}
};
\end{scope}

\end{scope}


\end{tikzpicture}

\end{document}

